%%%%%%%%%%%%%%%%%%%%%%%%%%%%%%%%%%%%%%%%%%%%%%%%%%%%%%%%%%%%%%%%%%%%%%%%%%%%%%
% CTCDevelopersGuide.tex
% Abstract: Documentation of CTC Code and Algorithms
%
% white paper
%
% Created: Chris L. Wyatt 5/22/2007
%%%%%%%%%%%%%%%%%%%%%%%%%%%%%%%%%%%%%%%%%%%%%%%%%%%%%%%%%%%%%%%%%%%%%%%%%%%%%%

\documentclass[12pt]{book}
\usepackage{pslatex}

\title{CTC Developers Guide}

\author{Christoper L Wyatt \\
  Assistant Professor \\
  Department of Electrical and Computer Engineering\\
  Virginia Tech\\
  Blacksburg, VA 24061-0111 \\
  clwyatt@vt.edu 
}

\date{May 2007}

\begin{document}
    
\maketitle

\tableofcontents

% ------------------------------------------------------------------------ %
% INTRODUCTION
%
\part{Introduction}
\chapter{Purpose}
This document describes the software library, applications, and
algorithms used in the CTC system.

\chapter{CT Colonography}
CT Colonography (aka Virtual Colonoscopy \cite{Vining1996} is a
minimally invasive screening procedure for colorectal polyps.

% ------------------------------------------------------------------------ %
% SEGEMENTATION
%
\part{Segmentation}
\chapter{Literature Review}

\chapter{Methods}

\chapter{Results}

% ------------------------------------------------------------------------ %
% REGISTRATION
%
\part{Registration}
\chapter{Literature Review}

\chapter{Methods}

\chapter{Results}

% ------------------------------------------------------------------------ %
% FEATURE EXTRACTION
%
\part{Feature Extraction}
\chapter{Literature Review}
\section{Yoshida and Nappi 2001}
Hiroyuki Yoshida and Janne Nappi, Three-Dimensional Computer-Aided
Diagnosis Scheme for Detection of Colonic Polyps, IEEE Transactions on
Medical Imaging, 2001 \cite{YoshidaN01}. These are the same features
used in \cite{YoshidaMMRD02}.

\par 
The features used are functions of the local principle
curvatures, $\kappa_1$ and $\kappa_2$, computed point-wise at $p$.
\begin{itemize}
\item {\it volumetric shape index}:
\[
SI(p) = \frac{1}{2}-\frac{1}{\pi}\arctan{\frac{\kappa_1(p)+\kappa_2(p)}{\kappa_1(p)-\kappa_2(p)}}
\]
\item {\it curvedness}:
\[
CV(p) = \sqrt{\frac{\kappa_1(p)^2 + \kappa_2(p)^2}{2}} 
\]
\end{itemize}

\section{Nappi and Yoshida 2002}
Janne Nappi and Hiroyuki Yoshida, Automated Detection of Polyps with
CT Colonography: Evaluation of Volumetric Features for Reduction of
False-Positive Findings, Academic Radiology, 2002 \cite{NappiY02}.
\par This paper introduces a feature for false positive reduction,
computed point-wise at $p$, called the
{\it directional gradient concentration} (DGC), defined as
\[
GC(p)=\frac{1}{N}\sum_{i=1}^{N} e_i^{\max}(p) \; ,
\]
where 
\[
e_i^{\max}(p) = \max_{R_{\min} \leq n \leq R_{\max}}
\left\{ 
\frac{1}{n-R_{\min}+1}\sum_{j=R_{\min}}^{n} \cos{\psi_{ij}(p)}
\right\} \; .
\]
$N$ is the number of directions, $D_i$, in which the surrounding
gradients are estimated. $\psi_{ij}$ is the angle between a direction
vector, and the gradient at a position of length $j$ from $p$ in the
direction $D_i$. $R_{\min}$ and $R_{\max}$ are the bounds on the
length $j$. Values for these constants are not given.

\par Note, that this is roughly the same as the
divergence of the gradient.  

\par A modified version of this this
feature was presented in \cite{NappiY03}, but showed no improvement.
\[
MGC(p) = \frac{1}{1+e^{g GC(p)-t)} \; ,}
\]
for scalars $g$ (the gain) and $t$ (the transfer). Values for these
constants are not given.

\chapter{Methods}


\chapter{Results}


% ---------------------------------------------------------------------- %
% REFERENCES
%
\clearpage
\bibliographystyle{plain}
\bibliography{CTC}
\addcontentsline{toc}{Chapter}{References}

\end{document}
